\documentclass[11pt]{article}

\usepackage{amsmath,amsthm,amssymb}
\usepackage{hyperref}
%\usepackage{geometry}
\usepackage{graphicx}
\usepackage{enumerate}
%\usepackage{caption,subcaption}



\theoremstyle{definition}
\newtheorem{thm}{Theorem}
\newtheorem{lem}[thm]{Lemma}
\newtheorem{prop}[thm]{Proposition}
\newtheorem{rem}[thm]{Remark}
\newtheorem{hyp}[thm]{Hypothesis}
\newtheorem{ex}[thm]{Example}

\numberwithin{equation}{section}
\numberwithin{thm}{section}

\DeclareMathOperator\erf{erf}


\renewcommand{\a}{a}
\renewcommand{\b}{b}
\newcommand{\m}{m}
%\renewcommand{\alpha}{{w_1^{-1}(\a)}}
%\renewcommand{\beta}{{w_1^{-1}(\b)}}
%\renewcommand{\alpha}{\alpha}
%\renewcommand{\beta}{\beta}

\usepackage{authblk}


\begin{document}
This document is to try to write all my examples down.

TODO:

\begin{enumerate}
\item Remove the example labels. It makes it look like a textbook.

\item Reduce the number of unneeded equations, and condense the discussion into paragraphs.

\item Include more figures.

\item Don't forget the condition on the maximum of $w_2$
\end{enumerate}

\section{Examples}

In this section, we construct the periodic traveling wave solution for several well-known disperal kernels in population biology, namely the uniform, Laplace, and normal distributions. For the uniform and Laplace kernels, we were able to construct a piecewise expression for the mean wave speed in terms of the model parameters. 

\begin{ex}
The Laplace kernel,
\begin{equation}\label{laplacekernel}
k(x) = \frac{1}{2} e^{-|x|}
\end{equation}

The reader can easily verify that the Laplace kernel satisfies hypotheses (H1) - (H3). The proof that is also satisfies (H4) is left in the appendix. The periodic traveling waves are given by

\begin{equation}
w_1(x) =   \begin{cases} 
1 - \frac{1}{2}e^{x} & x \leq 0 \\
\frac{1}{2}e^{-x} & x > 0
\end{cases}
\end{equation}
and
%\begin{equation}\label{laplacew2}
%w_2(x) = \begin{cases}
%m - \frac{1}{2} e^{x - \alpha} + \frac{1-\m}{2} e^{x - \beta} & x < \beta \\
%1 - \frac{1}{2} e^{x-\alpha} - \frac{1-\m}{2} e^{-(x-\beta)} & \beta \leq x \leq \alpha \\
%\frac{1}{2} e^{-(x-\alpha)} - \frac{1-\m}{2} e^{-(x-\beta)} & x > \alpha
%\end{cases}
%\end{equation}
\begin{equation}
w_2(x) = \begin{cases}
m + C_1 e^x & x < \beta \\
1 - C_2 e^x - C_3 e^{-x} & \beta < x < \alpha \\
C_4 e^{-x} & \alpha < x 
\end{cases} \end{equation}
where $\alpha=w_1^{-1}(\a)$, $\beta=w_1^{-1}(\b)$, with
$$ w_1^{-1}(p) = \begin{cases} -\ln(2p) & p\leq \frac{1}{2} \\ \ln(2-2p) & p > \frac{1}{2} \end{cases} $$
The constants $C_1,C_2,C_3$, and $C_4$ are continuous functions of the growth parameters, with $C_2,C_3,C_4\geq0$, and they are given by
\newcommand{\Conecaseone}{\b(1-\m)-\a}
\newcommand{\Conecasetwo}{\frac{1-\m-4\a(1-\b)}{4(1-\b)}}
\newcommand{\Conecasethree}{- \frac{1-\b+\m(1-\a)}{4(1-\a)(1-\b)} }

\begin{equation}
C_1 = \begin{cases}
\Conecaseone & \a,\b < \frac{1}{2} \\
\Conecasetwo & \a<\frac{1}{2}<\b \\
\Conecasethree & \frac{1}{2}<a,b
\end{cases} \end{equation}

\newcommand{\Ctwocaseone}{\a}
\newcommand{\Ctwocasetwo}{\frac{1}{4(1-\a)}}

\begin{equation}
C_2 = \begin{cases}
\Ctwocaseone & \a < \frac{1}{2} \\
\Ctwocasetwo & \a > \frac{1}{2}
\end{cases} \end{equation}

\newcommand{\Cthreecaseone}{\frac{1-\m}{4\b}}
\newcommand{\Cthreecasetwo}{(1-\m)(1-\b)}

\begin{equation}
C_3 = \begin{cases}
\Cthreecaseone & \b < \frac{1}{2} \\
\Cthreecasetwo & \b > \frac{1}{2}
\end{cases} \end{equation}

\newcommand{\Cfourcaseone}{\frac{\b-\a(1-m)}{4\a\b}}
\newcommand{\Cfourcasetwo}{\frac{1-4\a(1-\m)(1-\b)}{4\a}}
\newcommand{\Cfourcasethree}{1-\a-(1-\m)(1-\b)}

\begin{equation}
C_4 = \begin{cases}
\Cfourcaseone & \a,\b < \frac{1}{2} \\
\Cfourcasetwo & \a<\frac{1}{2}<\b \\
\Cfourcasethree & \frac{1}{2}<a,b
\end{cases} \end{equation}

To find $c^*$, we can now condition on the values of $w_2(\alpha)$ and $w_2(\beta)$.
% If $w_2(\alpha)=C_4e^{-\alpha}>a$, then we know $x_a>\alpha$, so that $C_4e^{-x_a}=C_4e^{-2c^*}=a$ implies $c^*=\frac{1}{2}\ln(\frac{C_4}{a})$.
%
%If $w_2(\alpha)<a$ and $w_2(\beta)>a$, then we solve the equation
%
%$$ 1 - C_2e^x - C_3e^{-x} = a $$
%
%Multiplying through by $e^x$ yields a quadratic in $e^x$:
%
%$$ e^x - C_2e^{2x} - C_3 = ae^x $$
%
%$$ C_2 e^{2x} + (a-1) e^x + C_3 = 0 $$
%
%$$ e^x = \frac{1-a \pm \sqrt{(1-a)^2 - 4C_2C_3}}{2C_2} $$
%
%Since $C_2,C_3$ are negative, we know the function is concave down on this piece. Therefore, we take the rightmost solution, since the curve must be increasing through this point since the solution is unique.
%
%$$ c^* = \frac{1}{2} \ln \left( \frac{1-a + \sqrt{(1-a)^2 - 4C_2C_3}}{2C_2} \right) $$
%
%In the final case where $w_2(\beta)<a$, we have
%
%$$ m + C_1 e^{x_a} = a $$
%implies
%$$ C_1e^{x_a} = a-m$$
%$$ x_a = \ln \frac{a-m}{C_1} $$
%$$ c^* = \frac{1}{2} \ln \frac{a-m}{C_1} $$
%
%Our general formula is thus
%\begin{equation}
%c^* = \begin{cases}
%\frac{1}{2} \ln \left( \frac{C_4}{a} \right) & a<w_2(\alpha) \\
%\frac{1}{2} \ln \left( \frac{1-a + \sqrt{(1-a)^2 - 4C_2C_3}}{2C_2} \right) & w_2(\alpha)<a<w_2(\beta) \\
%\frac{1}{2} \ln \left( \frac{a-m}{C_1}\right) & a>w_2(\beta)
%\end{cases}
%\end{equation}
%Equivalently,
\begin{equation}
c^* = \begin{cases}
\frac{1}{2} \ln \left( \frac{C_4}{a} \right) & a<C_4e^{-\alpha} \\
\frac{1}{2} \ln \left( \frac{1-a + \sqrt{(1-a)^2 - 4C_2C_3}}{2C_2} \right) & C_4e^{-\alpha}<a<m+C_1e^\alpha\\
\frac{1}{2} \ln \left( \frac{a-m}{C_1}\right) & a>m+C_1e^\alpha
\end{cases}
\end{equation}

Since the form of $w_2(x)$, and thus of $c^*$, depends on the values of $\a$ and $\b$; thus we will split into three cases for further analysis.

%Let $c_1$ and $c_2$ denote the intermediate wave speeds measured at some threshold $0<h<m$. For this analysis we chose the threshold of $h=a$. Thus $c_1 = w_2^{-1}(a) - w_1^{-1}(a)=2c^*-\alpha$, and $c_2=2c^*+\alpha-c_1=2\alpha$. We can now define the relative fluctuation in wavespeed by
%
%$$ f = \frac{|c_1-c_2|}{c^*} = \frac{|2c^*-3\alpha|}{c^*} $$
%
%We can also consider
%
%$$ f^2 = \frac{(c_1-c_2)^2}{(c^*)^{2}} $$
%
%The derivative is
%
%$$ \frac{df^2}{da} = \frac{d}{da}(c_1-c_2)^2\frac{1}{(c^*)^2} $$
%
%$$ = 2(c_1-c_2)(\frac{dc_1}{da}-\frac{dc_2}{da})(c^*)^{-2} - 2(c_1-c_2)^2(c^*)^{-3} $$
%
%$$ = \frac{2(c_1-c_2)}{(c^*)^2}\left((\frac{dc_1}{da}-\frac{dc_2}{da}) - \frac{c_1-c_2}{c^*}\right) $$
%
%It follows that $f$ is differentiable with respect to all the parameters almost everywhere. We conjecture that
%
%$$ \frac{df}{da} \geq 0 , \quad \text{almost everywhere} $$
%
%We will first consider the case $a,b<\frac{1}{2}$, and $a<C_4e^{-\alpha}$. We then have $\alpha=-\ln(2a)$, $c^*=\frac{1}{2}\ln\frac{b-(1-m)a}{4a^2b}$ so that
%
%$$ f = \frac{|\ln\frac{b-(1-m)a}{4a^2b}+3\ln(2a)|}{\ln\frac{b-(1-m)a}{4a^2b}} $$
%$$ = \frac{|\ln(2a(b-(1-m)a))-\ln(b)|}{\ln(b-(1-m)a)-\ln(4a^2b)}  $$

\begin{enumerate}[{Case} 1.]

\item $\a<\frac{1}{2}$, $\b<\frac{1}{2}$.
\begin{equation}
w_2(x) = \begin{cases}
\m + (\Conecaseone) e^x  & x < -\ln(2\b) \\
1 - \Ctwocaseone e^x - \Cthreecaseone e^{-x} & -\ln(2\b) < x < -\ln(2\a) \\
\Cfourcaseone e^{-x} & x > -\ln(2\a)
\end{cases}
\end{equation}

We have $w_2(\alpha)=\frac{\b-\a(1-\m)}{2\b}$ and $w_2(\beta)=\m + \frac{\b(1-\m)-\a}{2\b}$. Thus,
\begin{equation}
c^* = \begin{cases}
\frac{1}{2} \ln \left( \frac{\b-\a(1-\m)}{4\a^2\b} \right) & \a < \frac{\b-\a(1-\m)}{2\b} \\
\frac{1}{2} \ln \left( \frac{1-a + \sqrt{(1-a)^2 - \frac{\a(1-m)}{b}}}{2\a} \right) & \frac{\b-\a(1-\m)}{2\b} < a < \frac{\b(1+\m)-\a}{2\b} \\
\frac{1}{2} \ln \left( \frac{a-m}{\b(1-\m)-\a}\right) & a > \frac{\b(1+\m)-\a}{2\b}
\end{cases}
\end{equation}


\item $\a<\frac{1}{2}$, $\b>\frac{1}{2}$.
\begin{equation}
w_2(x) = \begin{cases}
\m + \Conecasetwo e^x  & x < \ln(2-2\b) \\
1 - \Ctwocaseone e^x - \Cthreecasetwo e^{-x} & \ln(2-2\b) < x < -\ln(2\a) \\
\Cfourcasetwo e^{-x} & x > -\ln(2\a)
\end{cases}
\end{equation}

We have $w_2(\alpha)=\frac{1-4\a(1-\m)(1-\b)}{2}$ and $w_2(\beta)=\frac{1+\m-4\a(1-\b)}{2}$. Thus,
\begin{equation}
c^* = \begin{cases}
\frac{1}{2} \ln \left( \frac{1-4\a(1-\m)(1-\b)}{4\a^2} \right) & a<\frac{1-4\a(1-\m)(1-\b)}{2} \\
\frac{1}{2} \ln \left( \frac{1-a + \sqrt{(1-a)^2 - 4a(1-\m)(1-\b)}}{2a} \right) & \frac{1-4\a(1-\m)(1-\b)}{2}<a<\frac{1+\m-4\a(1-\b)}{2} \\
\frac{1}{2} \ln \left( \frac{4(a-m)(1-b)}{1-\m-4\a(1-\b)}\right) & a>\frac{1+\m-4\a(1-\b)}{2}
\end{cases}
\end{equation}


\item $\a>\frac{1}{2}$, $\b>\frac{1}{2}$.
\begin{equation}
w_2(x) = \begin{cases}
\m \Conecasethree e^x  & x < \ln(2-2\b) \\
1 - \Ctwocasetwo e^x - \Cthreecasetwo e^{-x} & \ln(2-2\b) < x < \ln(2-2\a) \\
(\Cfourcasethree) e^{-x} & x > \ln(2-2\a)
\end{cases}
\end{equation}

Thus, $w_2(\alpha)=\frac{1-\a-(1-\m)(1-\b)}{2(1-a)}$ and $w_2(\beta)=\m - \frac{1-\b+\m(1-\a)}{2(1-\a)}$

\begin{equation}
c^* = \begin{cases}
\frac{1}{2} \ln \left( \frac{1-\a-(1-\m)(1-\b)}{a} \right) & a<\frac{1-\a-(1-\m)(1-\b)}{2(1-a)} \\
\frac{1}{2} \ln \Big( 2(1-a)\left[1-a + \sqrt{\frac{(1-a)^3-(1-\m)(1-\b) }{1-\a}}\right] \Big) & \frac{1-\a-(1-\m)(1-\b)}{2(1-a)}<a<\frac{\b-1+\m(1-\a)}{2(1-\a)} \\
\frac{1}{2} \ln \left( \frac{4(m-a)(1-\a)(1-\b)}{1-\b+\m(1-\a)}\right) & a>\frac{\b-1+\m(1-\a)}{2(1-\a)}
\end{cases}
\end{equation}

\end{enumerate}

%%%%%%%%%%%%%%%%%%%%%%%%%%%%%%%%%%%%%%%%%%%%%%%%
%They are given explicitly by
%
%\renewcommand{\Conecaseone}{\b\m'-\a}
%\renewcommand{\Conecasetwo}{\frac{\m'-4\a\b'}{4\b'}}
%\renewcommand{\Conecasethree}{- \frac{\b'+\m\a'}{4\a'\b'} }
%
%\begin{equation}
%C_1 = \begin{cases}
%\Conecaseone & \a,\b < \frac{1}{2} \\
%\Conecasetwo & \a<\frac{1}{2}<\b \\
%\Conecasethree & \frac{1}{2}<a,b
%\end{cases} \end{equation}
%
%\renewcommand{\Ctwocaseone}{\a}
%\renewcommand{\Ctwocasetwo}{\frac{1}{4\a'}}
%
%\begin{equation}
%C_2 = \begin{cases}
%\Ctwocaseone & \a < \frac{1}{2} \\
%\Ctwocasetwo & \a > \frac{1}{2}
%\end{cases} \end{equation}
%
%\renewcommand{\Cthreecaseone}{\frac{\m'}{4\b}}
%\renewcommand{\Cthreecasetwo}{\m'\b'}
%
%\begin{equation}
%C_3 = \begin{cases}
%\Cthreecaseone & \b < \frac{1}{2} \\
%\Cthreecasetwo & \b > \frac{1}{2}
%\end{cases} \end{equation}
%
%\renewcommand{\Cfourcaseone}{\frac{\b-\a\m'}{4\a\b}}
%\renewcommand{\Cfourcasetwo}{\frac{1-4\a\m'\b'}{4\a}}
%\renewcommand{\Cfourcasethree}{\a'-\m'\b'}
%
%\begin{equation}
%C_4 = \begin{cases}
%\Cfourcaseone & \a,\b < \frac{1}{2} \\
%\Cfourcasetwo & \a<\frac{1}{2}<\b \\
%\Cfourcasethree & \frac{1}{2}<a,b
%\end{cases} \end{equation}
%
%To find $c^*$, we can now condition on the values of $w_2(\alpha)$ and $w_2(\beta)$.
% If $w_2(\alpha)=C_4e^{-\alpha}>a$, then we know $x_a>\alpha$, so that $C_4e^{-x_a}=C_4e^{-2c^*}=a$ implies $c^*=\frac{1}{2}\ln(\frac{C_4}{a})$.
%
%If $w_2(\alpha)<a$ and $w_2(\beta)>a$, then we solve the equation
%
%$$ 1 - C_2e^x - C_3e^{-x} = a $$
%
%Multiplying through by $e^x$ yields a quadratic in $e^x$:
%
%$$ e^x - C_2e^{2x} - C_3 = ae^x $$
%
%$$ C_2 e^{2x} + (a-1) e^x + C_3 = 0 $$
%
%$$ e^x = \frac{1-a \pm \sqrt{(1-a)^2 - 4C_2C_3}}{2C_2} $$
%
%Since $C_2,C_3$ are negative, we know the function is concave down on this piece. Therefore, we take the rightmost solution, since the curve must be increasing through this point since the solution is unique.
%
%$$ c^* = \frac{1}{2} \ln \left( \frac{1-a + \sqrt{(1-a)^2 - 4C_2C_3}}{2C_2} \right) $$
%
%In the final case where $w_2(\beta)<a$, we have
%
%$$ m + C_1 e^{x_a} = a $$
%implies
%$$ C_1e^{x_a} = a-m$$
%$$ x_a = \ln \frac{a-m}{C_1} $$
%$$ c^* = \frac{1}{2} \ln \frac{a-m}{C_1} $$
%
%%Our general formula is thus
%\begin{equation}
%c^* = \begin{cases}
%\frac{1}{2} \ln \left( \frac{C_4}{a} \right) & a<w_2(\alpha) \\
%\frac{1}{2} \ln \left( \frac{\a' + \sqrt{\a'^2 - 4C_2C_3}}{2C_2} \right) & w_2(\alpha)<a<w_2(\beta) \\
%\frac{1}{2} \ln \left( \frac{a-m}{C_1}\right) & a>w_2(\beta)
%\end{cases}
%\end{equation}
%
%Since the form of $w_2(x)$, and thus of $c^*$, depends on the values of $\a$ and $\b$; thus we will split into three cases for further analysis.
%
%\begin{enumerate}[{Case} 1.]
%
%\item $\a<\frac{1}{2}$, $\b<\frac{1}{2}$.
%\begin{equation}
%w_2(x) = \begin{cases}
%\m + (\Conecaseone) e^x  & x < -\ln(2\b) \\
%1 - \Ctwocaseone e^x - \Cthreecaseone e^{-x} & -\ln(2\b) < x < -\ln(2\a) \\
%\Cfourcaseone e^{-x} & x > -\ln(2\a)
%\end{cases}
%\end{equation}
%
%We have $w_2(\alpha)=\frac{\b-\a\m'}{2\b}$ and $w_2(\beta)=\m + \frac{\b\m'-\a}{2\b}$. Thus,
%\begin{equation}
%c^* = \begin{cases}
%\frac{1}{2} \ln \left( \frac{\b-\a\m'}{4\a^2\b} \right) & \a < \frac{\b-\a\m'}{2\b} \\
%\frac{1}{2} \ln \left( \frac{\a'+ \sqrt{\a'^2 - \frac{\a\m'}{b}}}{2\a} \right) & \frac{\b-\a\m'}{2\b} < a < \m + \frac{\b\m'-\a}{2\b} \\
%\frac{1}{2} \ln \left( \frac{a-m}{\b\m'-\a}\right) & a > \m + \frac{\b\m'-\a}{2\b}
%\end{cases}
%\end{equation}
%
%
%\item $\a<\frac{1}{2}$, $\b>\frac{1}{2}$.
%\begin{equation}
%w_2(x) = \begin{cases}
%\m + \Conecasetwo e^x  & x < \ln(2\b') \\
%1 - \Ctwocaseone e^x - \Cthreecasetwo e^{-x} & \ln(2\b') < x < -\ln(2\a) \\
%\Cfourcasetwo e^{-x} & x > -\ln(2\a)
%\end{cases}
%\end{equation}
%
%We have $w_2(\alpha)=\frac{1-4\a\b'\m'}{2}$ and $w_2(\beta)=\frac{1+\m-4\a\b'}{2}$. Thus,
%\begin{equation}
%c^* = \begin{cases}
%\frac{1}{2} \ln \left( \frac{1-4\a\b'\m'}{4\a^2} \right) & a<\frac{1-4\a\b'\m'}{2} \\
%\frac{1}{2} \ln \left( \frac{\a' + \sqrt{\a'^2 - 4\a\b'\m'}}{2\a} \right) & \frac{1-4\a\b'\m'}{2}<a<\frac{1+\m-4\a\b'}{2} \\
%\frac{1}{2} \ln \left( \frac{4\b'(a-m))}{\m'-4\a\b'}\right) & a>\frac{1+\m-4\a\b'}{2}
%\end{cases}
%\end{equation}
%
%
%\item $\a>\frac{1}{2}$, $\b>\frac{1}{2}$.
%\begin{equation}
%w_2(x) = \begin{cases}
%\m \Conecasethree e^x  & x < \ln(2\b') \\
%1 - \Ctwocasetwo e^x - \Cthreecasetwo e^{-x} & \ln(2\b') < x < \ln(2\a') \\
%(\Cfourcasethree) e^{-x} & x > \ln(2\a')
%\end{cases}
%\end{equation}
%
%Thus, $w_2(\alpha)=\frac{\a'-\b'\m'}{2\a'}$ and $w_2(\beta)=\m - \frac{\b'+\a'\m}{2\a'}$
%
%\begin{equation}
%c^* = \begin{cases}
%\frac{1}{2} \ln \left( \frac{\a'-\b'\m'}{a} \right) & a<\frac{\a'-\b'\m'}{2\a'} \\
%\frac{1}{2} \ln \left( 2\a'^2 + 2\a'\sqrt{\a'^2 - \frac{\b'\m'}{\a'}} \right) & \frac{\a'-\b'\m'}{2\a'}<a<\m - \frac{\b'+\a'\m}{2\a'} \\
%\frac{1}{2} \ln \left( \frac{4\a'\b'(m-a)}{\b'+\a'\m}\right) & a>\m - \frac{\b'+\a'\m}{2\a'}
%\end{cases}
%\end{equation}
%
%\end{enumerate}
%%%%%%%%%%%%%%%%%%%%%%%%%%%%%%%%%%%%%%%%%%%%%%%%%%
%
%Each of the preceding cases can be generalized as follows:
%
%\begin{equation}
%w_2(x) = \begin{cases}
%m + C_1 e^x & x < \beta \\
%1 - C_2 e^x - C_3 e^{-x} & \beta < x < \alpha \\
%C_4 e^{-x} & \alpha < x 
%\end{cases} \end{equation}
%with 
%
%Note that $C_2$ and $C_3$ are strictly negative. 
%
%To find $c^*$, we can now condition on the values of $w_2(\alpha)$ and $w_2(\beta)$. If $w_2(\alpha)=C_4e^{-\alpha}>a$, then we know $x_a>\alpha$, so that $C_4e^{-x_a}=C_4e^{-2c^*}=a$ implies $c^*=\frac{1}{2}\ln(\frac{C_4}{a})$.
%
%If $w_2(\alpha)<a$ and $w_2(\beta)>a$, then we solve the equation
%
%$$ 1 - C_2e^x - C_3e^{-x} = a $$
%
%Multiplying through by $e^x$ yields a quadratic in $e^x$:
%
%$$ e^x - C_2e^{2x} - C_3 = ae^x $$
%
%$$ C_2 e^{2x} + (a-1) e^x + C_3 = 0 $$
%
%$$ e^x = \frac{1-a \pm \sqrt{(1-a)^2 - 4C_2C_3}}{2C_2} $$
%
%Since $C_2,C_3$ are negative, we know the function is concave down on this piece. Therefore, we take the rightmost solution, since the curve must be increasing through this point since the solution is unique.
%
%$$ c^* = \frac{1}{2} \ln \left( \frac{1-a + \sqrt{(1-a)^2 - 4C_2C_3}}{2C_2} \right) $$
%
%In the final case where $w_2(\beta)<a$, we have
%
%$$ m + C_1 e^{x_a} = a $$
%implies
%$$ C_1e^{x_a} = a-m$$
%$$ x_a = \ln \frac{a-m}{C_1} $$
%$$ c^* = \frac{1}{2} \ln \frac{a-m}{C_1} $$
%
%Our general formula is thus
%\begin{equation}
%c^* = \begin{cases}
%\frac{1}{2} \ln \left( \frac{C_4}{a} \right) & a<w_2(\alpha) \\
%\frac{1}{2} \ln \left( \frac{1-a + \sqrt{(1-a)^2 - 4C_2C_3}}{2C_2} \right) & w_2(\alpha)<a<w_2(\beta) \\
%\frac{1}{2} \ln \left( \frac{a-m}{C_1}\right) & a>w_2(\beta)
%\end{cases}
%\end{equation}
%
%Let $x_a=w_2^{-1}(a)$ (well-defined by Lemma 1.2). We can check which piece $x_a$ lies on by evaluating $w_2(x)$ at $x=\alpha$ and $x=\beta$.
%
%
%Since $w_2$ is increasing for $x<\log(2a)$, we know the solution to $w_2(x)=a$ must lie on the second or third pieces. This can be determined by checking the sign of $w_2(\log(2b))-a$. If it is positive, the horizontal line $y=a$ intersects the curve $y=w_2(x)$ on the third piece, and the solution can be found by solving $(mb+b-a)e^{-x}=a$. Otherwise, if the sign is negative, the solution occurs on the second piece as the solution to the equation
%
%$$ w_2(x)=a \iff m + 1 - \frac{m+1}{4b} e^x - ae^{-x} = a $$
%
%First let us group like terms and multiply by $4b$:
%
%$$ (m+1) e^x + (m + 1 - a) - 4abe^{-x} = 0 $$
%
%Using the substitution $y=e^x$, this equation becomes a quadratic in $y$:
%
%$$ (m+1) y^2 + (m + 1 - a) y - 4ab = 0 $$
%
%The solution is
%
%$$ y = \frac{a-m-1 \pm \sqrt{(m+1-a)^2 + 16ab(m+1)}}{2m+2}$$
%
%We know the global maximum of $w_2(x)$ lies somewhere in the open interval $(\log(2a),\log(2b))$, therefore we take the greater solution so that
%
%$$ c^* = \frac{1}{2} \log\left(  \frac{a-m-1 + \sqrt{(m+1-a)^2 + 16ab(m+1)}}{2m+2} \right) $$
%
%Then we can obtain the critical wave speed in two cases:
%$$ c^* = \begin{cases} \displaystyle
%\frac{1}{2} \log \left( \frac{mb+b-a}{a} \right) & m \geq \frac{2ab+a-b}{b} \\
%\displaystyle
% \frac{1}{2} \log\left(  \frac{a-m-1 + \sqrt{(m+1-a)^2 + 16ab(m+1)}}{2m+2} \right) & m < \frac{2ab+a-b}{b}
%\end{cases} $$
%
%Case 2: $a<\frac{1}{2}<b$. Then
%$$ w_2(x) = \int_{x+\log(2-2b)}^{\infty} \frac{m}{2}e^{-|y|}\,dy + \int_{x+\log(2-2b)}^{x-\log(2a)} \frac{1}{2}e^{-|y|} \,dy $$
%
%Then
%
%$$ w_2(x) = \frac{m}{2} - \frac{m+1}{2}\left( \min(\frac{e^{-x}}{2-2b},1) - \min((2-2b)e^{x},1) \right) - \frac{1}{2} \left( \min(2ae^{-x},1) - \min(\frac{1}{2a}e^{x},1) \right) $$
%
%so
%
%$$ w_2(x) = \begin{cases}
%\displaystyle
%m - \frac{4a(m+1)(1-b)-1}{4a} e^x & x <\log(2a) \\
%\displaystyle
%m+1 - (m+1)(a-b) e^x - ae^{-x} & \log(2a) \leq x \leq -\log(2-2b) \\
%\displaystyle
%\frac{m+1-4a(1-b)}{4(1-b)} e^{-x}  & x > -\log(2-2b)
%\end{cases} $$

\end{ex}

\begin{ex} Consider the Gaussian kernel with zero mean and unit variance given by
$$ k(x) = \frac{1}{\sqrt{2\pi}} e^{-\frac{x^2}{2}} $$
The kernel is symmetric and has connected support, hence it satisfies hypotheses (H1)-(H3); the proof for hypothesis (H4) is left in the appendix.

Let $\Phi(x)=\int_{-\infty}^{x}k(y)\,dy$ denote the cumulative density function of the standard normal distribution, and $\Phi^{-1}$ be its inverse. The periodic traveling wave solutions $w_1(x)$ and $w_2(x)$ are given by
\begin{equation}
w_1(x) = \Phi(-x)
\end{equation}
and
\begin{equation}
w_2(x)=  m - \Phi(x-\Phi^{-1}(a)) + (1-m)\Phi(x-\Phi^{-1}(b))
\end{equation}
where $\alpha=\Phi^{-1}(a)$ and $\beta=\Phi^{-1}(b)$.

$w_2$ has a unique global maximum at $x^*=\frac{\alpha+\beta}{2} + \frac{1}{\alpha-\beta}\ln\left(1-m\right)$. Thus, by Theorem \ref{theorem1}, $w_1$ and $w_2$ are a periodic traveling wave solution if $w_2(x^*)\leq b$.
\end{ex}

%\begin{ex} The Laplace kernel is given by
%$$ k(x) = \frac{1}{2} e^{-\frac{1}{2}|x|} $$
%
%It can be checked that $k(x)$ satisfies all of Hypothesis \ref{hypothesis1}. To show part iv., let $\mu\in(0,1)$, and assume without loss of generality $y>0$. (If $y=0$ then $f(x)$ is simply $k(x)$ rescaled, hence strictly positive.) For $y>0$, we have
%$$ f(x) = k(x) - \mu k(x-y) = \frac{1}{2} \begin{cases}
%e^x - \mu e^{x-y} & \text{if } x < 0 \\
%e^{-x} - \mu e^{x-y} & \text{if } 0 \leq x < y \\
%
%e^{-x} -\mu e^{-(x-y)} & \text{if } x \geq y
%\end{cases} $$
%$f$ is continuous, satisfies $f(-\infty)=f(\infty)=0$, is increasing on $(-\infty,0)$ and decreasing on $(0,\alpha)$. If $e^y\geq \mu$, then $f$ is decreasing on $(y,\infty)$, so $f$ has no zero-crossings Otherwise, if $e^y< \mu$, then $f$ is increasing on $(y,\infty)$ and has exactly one zero-crossing. In both cases, condition iv. is satisfied.
%
%We have three cases:
%
% If $a<\frac{1}{2}$ and $b<\frac{1}{2}$, then
%$$ \begin{aligned}
%w_2(-x) &= n_{2}K\left(x-\frac{1}{\alpha}\ln\left(2a\right)\right)-\left(1-m\right)K\left(x-\frac{1}{\alpha}\ln\left(2b\right)\right) \\
%&= \begin{cases}
%\left(\frac{1}{4a}-\frac{1-m}{4b}\right)e^{\alpha x} & x < \frac{1}{\alpha}\ln\left(2a\right) \\
%1-ae^{-\alpha x}-\frac{1-m}{4b}e^{\alpha x} &  \frac{1}{\alpha}\ln\left(2a\right) \leq x < \frac{1}{\alpha}\ln\left(2b\right) \\
%m+\left((1-m)b-a\right)e^{-\alpha x} &  x \geq \frac{1}{\alpha}\ln\left(2b\right)
%\end{cases} 
%\end{aligned} $$
%If $b-(1-m)a\geq 2ab$, then $c^*$ is given by $c^*=\frac{1}{2\alpha}\ln\left(\frac{b-(1-m)a}{4a^2b}\right)$.
%\end{ex}

\begin{ex}
Consider the uniform dispersal kernel given by
\begin{equation}
k(x) = \begin{cases}
\frac{1}{2} & |x|\leq 1 \\
0 & |x| > 1
\end{cases} \end{equation}

Then $w_1$ is given by
\begin{equation}  \label{w1}
\begin{aligned}
w_1(x) 
= \begin{cases}
1, & x \in (-\infty, -1), \\
\frac{1}{2}-\frac{1}{2}x, & x \in [-1, 1] ,\\
0, & x \in (1, \infty),
\end{cases}
\end{aligned} \end{equation}
with inverse $w_1^{-1}(p)=1-2p$ for $0<p<1$. Let $\alpha=1-2a$ and $\beta=1-2b$. Then
\begin{equation} \label{w2}
\begin{aligned}
w_2(x) 
= \begin{cases}
m,
& x \in (-\infty, \beta-1), \\
\frac{1-m}{2}x + m + b - mb,
& x \in [\beta - 1, 
 \alpha- 1), \\
-\frac{m}{2}x +m+b- mb-a,
& x \in [\alpha - 1, \beta + 1), \\
-\frac{1}{2} x-a+1,
& x \in [\beta + 1, \alpha + 1], \\
0,
& x \in (\alpha+1,\infty).
\end{cases}
\end{aligned} \end{equation}

Observe that $w_2$ has a global maximum at $x=\alpha-1$ so that $||w_2||_\infty=w_2(\alpha-1)=m+(b-a)(1-m)$. By Theorem \ref{theorem2}, the pair $w_1$ and $w_2$ are a solution to equation \eqref{ptw} if $m-a<m(b-a)$.

We can also explicitly calculate the speed of the wave given by
\begin{equation} \label{c}
c^* = \begin{cases}
1 - 2a & \text{if } a \leq b/2, \\
1 -b + \frac{b - 2a}{m} & \text{if }a > b/2.
\end{cases}
\end{equation}
\end{ex}
\begin{rem}
$w_1(x)$ is positive for $x<1$ and zero for $x\geq1$, and $w_2(x)$ is positive for $x<2-2a$ and zero for $x\geq 2-2a$. Thus, $\eqref{q}$ has a traveling wave with wave profiles $w_1(x)$ and $w_2(x)$, intermediate wave speeds $c_1=1-2a$ and $c_2=2c^*-c_1$,  and average wave speed  $c^*$.  It is easily seen that $c_1=c_2$ if $a \leq b/2$, and $|c_1-c_2|=(2\alpha-\beta)(1-\frac{1}{m})>0$ if $a>b/2$. So for $a>b/2$, the traveling wave is periodic with two different intermediate wave speeds. Furthermore, the difference between these two intermediate speeds is increasing in $a$, decreasing in $b$, and increasing in $m$. This behavior is illustrated with two difference choices of parameters in Figure \ref{fig:wavespeed}.
\end{rem}


The regions in the parameter space where oscillating spreading speed exists can be determined as follows: for any fixed choice of $(n_1,n_2)$, with $0<n_1<n_2$, let $R$ be the set of pairs $(a,b)\in\mathbf R^2$ such that the hypothesis of Theorem 2.1 holds. Then $R$ is a triangle in the $a$-$b$ plane with endpoints at $(0,n_2)$, $(n_1,n_1)$, and $(n_1,n_2)$, depicted in Figure \ref{fig:phaseportrait}. The line $b=2a$ partitions $R$ into two non-empty sets $R_1=\{(a,b)\in R:a\leq b/2\}$ and $R_2=\{(a,b)\in R:a> b/2\}$ such that the traveling has constant speed if $(a,b)\in R_1$ and oscillating speed if $(a,b)\in R_2$.




\section{Appendix}

\begin{lem}
The Laplace kernel \eqref{laplacekernel} satisfies hypothesis (H4).
\end{lem}

\begin{proof}
let $f=f_{m,y}$ be the scalar function of $x$ with two parameters $y\in\mathbb R$ and $\mu\in(0,1)$ defined by
$$ f(x)= f_{m,y}(x) = \frac{1}{2} e^{-|x|} - \frac{\mu}{2}e^{-|x-y|} $$

If $y=0$, then $f$ has no zero-crossings, since $f_{m,0}(x)=\frac{1-\mu}{2}e^{-|x|}$ is strictly positive. If $y$ is nonzero, then one can easily check the symmetry relation $f_{m,-y}(x)=f_{m,y}(-x)$. Since the number of zero-crossings are invariant with respect to a reflection about the vertical axis, we can assume without loss of generality $y>0$.

Under this assumption, $f$ is strictly increasing on $(-\infty,0)$, and strictly decreasing on $(0,y)$. The behavior on $(y,\infty)$ is determined by the sign of $e^{-y}-m$. There are three cases:

\begin{enumerate}
\item if $y<\ln\frac{1}{m}$, then $f$ is decreasing on $(0,\infty)$, hence has no zero-crossings;
\item if $y>\ln\frac{1}{m}$, then $f$ has a unique zero-crossing at $x=\frac{1}{2}(y-\ln(m))$;
\item if $y=\ln\frac{1}{m}$, then $f$ vanishes on $(y,\infty)$, hence it has no zero-crossings.
\end{enumerate}
In each case, the number of zero-crossings does not exceed one.
\end{proof}

\begin{lem}
If $k(x)$ is given by the Laplace kernel, then $w_1$ and $w_2$ form a periodic traveling wave solution if $C_1\leq 0$, or if $C_1>0$ and $w_2\left(\ln\sqrt\frac{C_3}{C_2}\right)\leq b$.
\end{lem}

\begin{proof}
We can proceed in cases. If $C_1\leq 0$, then $w_2(x)$ is monotone decreasing, hence $w_2(x)<w_2(-\infty)=m<b$ everywhere. Otherwise, if $C_1>0$, then $w_2(x)$ is increasing on $(-\infty,\beta)$ and decreasing on $(\alpha,\infty)$. Since $w_2(x)$ is concave-down on $(\beta,\alpha)$, this implies there is a unique global maximum somewhere in this interval. To find it, we can differentiate:
$$ \frac{dw_2}{dx}\Big|_{\beta<x<\alpha}=C_3e^{-x}-C_2e^x $$
Setting this expression equal to zero and multiplying by $e^x$, we obtain $C_3-C_2e^{2x}=0$, which has a unique solution at $x=\ln\sqrt\frac{C_3}{C_2}$.
\end{proof}

\begin{lem}
The Gaussian kernel satisfies hypothesis H4.
\end{lem}

\begin{proof}
Let $y\in\mathbb R$ and $\mu\in(0,1)$. Then
\begin{equation} \begin{aligned}
k(x)-\mu k(x-y) &= \frac{1}{\sqrt{2\pi}} \left( e^{-\frac{x^2}{2}} - \mu e^{-\frac{-(x-y)^2}{2}} \right) \\
&= \frac{1}{\sqrt{2\pi}} e^{-\frac{x^2}{2}} \left( 1 - \mu e^{\frac{2xy-y^2}{2}} \right)
\end{aligned} \end{equation}
This expression has a unique zero at $x=\frac{y^2-2\ln(\mu)}{2y}$, so the number of zero-crossings is at most one.
\end{proof}

\begin{lem}
For the Gaussian kernel, $w_2(x)$ has a unique local extrema which is a global maxium at $x=\frac{2\ln(1-m)}{\alpha-\beta}+\alpha+\beta$.
\end{lem}

\begin{proof}
The derivative of $w_2(x)$ is given by
$$ 
\frac{dw_2}{dx} = -\frac{1}{\sqrt{2\pi}}e^{-\frac{(x-\alpha)^2}{2}} + \frac{1-m}{\sqrt{2\pi}}e^{-\frac{(x-\beta)^2}{2}}
$$
Setting this quantity equal to zero, we obtain the equation
$$ 
e^{-\frac{(x-\alpha)^2}{2}} = (1-m)e^{-\frac{(x-\beta)^2}{2}}
$$
Taking logarithm on both sides, and rearrange terms,
$$ 
(x-\beta)^2 = 2\ln(1-m) + (x-\alpha)^2
$$
Distributing both sides and cancelling the quadratic term, we get the solution
$$
x = \frac{2\ln(1-m)}{\alpha-\beta} + \alpha + \beta
$$
\end{proof}




\end{document}
