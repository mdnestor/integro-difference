\documentclass[11pt]{article}

\usepackage{amsmath,amsthm,amssymb}
\usepackage{hyperref}
%\usepackage{geometry}
\usepackage{graphicx}
\usepackage{enumerate}
%\usepackage{caption,subcaption}



\theoremstyle{definition}
\newtheorem{thm}{Theorem}
\newtheorem{lem}[thm]{Lemma}
\newtheorem{prop}[thm]{Proposition}
\newtheorem{rem}[thm]{Remark}
\newtheorem{hyp}[thm]{Hypothesis}
\newtheorem{ex}[thm]{Example}

\numberwithin{equation}{section}
\numberwithin{thm}{section}

\DeclareMathOperator\erf{erf}



\newcommand{\winva}{{w_1^{-1}(a)}}
\newcommand{\winvb}{{w_1^{-1}(b)}}


\usepackage{authblk}

\title{Periodic Traveling Waves in an Integro-Difference Equation With a Nonmonotone Growth Function and Strong Allee Effect}

\author{Michael Nestor, Bingtuan Li
\thanks{M. Nestor's email is \href{mailto:mdnest01@louisville.edu}{mdnest01@louisville.edu}. B. Li was partially supported by the National Science Foundation under Grant DMS-1515875 and Grant DMS-1951482.}}

\affil{Department of Mathematics, University of Louisville, \newline Louisville, KY 40292.}


\begin{document}


\maketitle


\begin{abstract}
We derive sufficient conditions for the existence of periodic traveling wave solutions for a class of integro-difference equation with piecewise constant growth function exhibiting a period two cycle and a strong Allee effect. We also prove the convergence of solutions with compactly supported initial data to translations of the traveling wave under appropriate conditions. 
\end{abstract}


{\bf Key words:} Integro-difference equation, period two cycle, Allee effect, periodic traveling wave.
\newline

{\bf AMS Subject Classification:} 92D40, 92D25.


\section{Introduction}

Integro-difference equations are of great interest in the studies of invasions of populations with discrete generations and separate growth and dispersal stages. They have been used to predict changes in gene frequency \cite{lui82a, lui82b, lui83, slatkin, w78}, and applied to ecological problems~\cite{hh, ks, kot89, kot92, kotbook, lut, nkl,otto}. Previous rigrous studies on integro-difference equations have assumed that the growth function is nondecreasing~\cite{w78, wein82}, or  is nonmonotone without strong Allee effect~\cite{lui83, wang}. The results show existence of constant spreading speeds and travelng waves with fixed shapes and speeds.   Sullivan et al. ~\cite{pnas} demonstrated numerically  that an integro-difference equation with a nonmonotone growth function exhibiting  a strong Allee effect can generate traveling waves with fluctuating speeds. In this paper we give a sufficient condition for the existence of periodic traveling waves with a periodic speed for such an equation with a specific growth function.


We consider the following integro-difference equation
\begin{align}\label{q}
u_{n+1}(x)\;=\;Q[u_n](x)\;:=(k*(g\circ u_n))(x)\;=\,\int^{\infty}_{-\infty}k(x-y)\,g\big(u_n(y)\big)\,\mathrm{d}y,
\end{align}
where 
\begin{equation} \label{g}
g(u) = \begin{cases}
0, & \text{if } u < a, \\
1, & \text{if } a \leq u \leq b, \\
m, & \text{if } u > b,
\end{cases}
\end{equation}
with $0<a<m<b<1$. $g(u)$ is a piecewise constant nonmonotone growth function exhibiting a strong Allee effect~\cite{all}. Specifically, it has a stable fixed point at zero and a stable period two cycle $(1,m)$ with $a$ the Allee threshold value.

Piecewise constant growth functions and uniform distributions have been used in the studies of integro-difference equations; see for example~\cite{kot1, lut,otto,  pnas}. We rigorously construct periodic traveling waves with periodic speeds for (\ref{q}). To the best of our knowledge, this is the first time that traveling waves with oscillating speeds have been analytically established  for scalar spatiotemporal equations with constant parameters. We also show the convergence of solutions with compactly supported initial data to translations of the traveling wave under appropriate conditions. Equation (\ref{q}) may be viewed as a symbolic model for integro-difference equations with a growth function exhibiting a strong Allee effect and a period two cycle. The results obtained this paper provide important insights into integro-difference equations with general growth functions. 


\section{Periodic traveling waves}

Let $C(\mathbb R)$ denote the space of continuous functions $\mathbb R\to\mathbb R$ equipped with the supremum-norm $||u||_\infty=\sup_{x\in\mathbb R}|u(x)|$ for any $u\in C(\mathbb R)$.

For the remainder of the paper, we assume the following hypothesis on the dispersal kernel:

\def\Hone{(\text{H1})}
\def\Htwo{(\text{H2})}
\def\Hthree{(\text{H3})}
\def\Hfour{(\text{H4})}

\begin{hyp} \label{hypothesis1} The disperal kernel $k$ is Lebesgue measurable, and

\begin{enumerate}[i.)]
\item $k$ is non-negative, and $\int_{-\infty}^{\infty} k(x) \, dx = 1$;

\item $k(x)=k(-x)$ for all $x\in\mathbb R$;

\item the support of $k$ is connected;

\item  for all $y\in\mathbb R$, for all $\mu\in(0,1)$, the function $f(x)= k(x)-\mu k(x-y)$ has at most one zero-crossing on $\mathbb R$.
\end{enumerate}
\end{hyp}

 Let $w_1$ and $w_2$ be two functions belonging to $C(\mathbb R)$ defined by
\begin{equation} \label{w1}
w_1(x) = \int_x^\infty k(y) \, dy
\end{equation}
and
\begin{equation} \label{w2}
w_2(x) = \int_{-\infty}^{\infty} k(y) g(w_1(x-y)) \, dy 
\end{equation}

Assuimption i. implies $w_1$ and $w_2$ have well-defined limits at $\pm\infty$ given by $w_1(\infty)=w_2(\infty)=0$, $w_1(-\infty)=1$, and $w_2(-\infty)=m$. Furthermore, $w_1$ is monotonically decreasing, while $w_2$ may be non-monotonic. Assumption iii. guarantees a unique right-inverse satisfying $w_1(w_1^{-1}(p))=p$ for $0<p<1$.

\begin{lem} \label{lemma1}
If condition (x) is satisfied then $Q[w_2](x)=w_1(x-2c^*)$ for a unique $c^*\in\mathbb R$.
\end{lem}

\begin{proof} Let $\alpha,\beta\in\mathbb R$ such that $w_1(\alpha)=a$ and $w_1(\beta)=b$.  By the definition of $w_1$, it is decreasing and satisfies $w_1(-\infty)=1$ and $w_1(\infty)=0$, hence $\alpha$ and $\beta$ are unique.  Applying the growth function $g$ yields
\begin{equation} \label{gw1}
g(w_1(x)) = \begin{cases}
m & x < \beta \\
1 & \beta \leq x \leq \alpha \\
0 & x > \alpha
\end{cases} \end{equation}
Applying the convolution operator yields a formula for $w_2(x)$:
\begin{equation} \label{w2}
\begin{aligned}
w_2(x) &= \int_{-\infty}^{\infty} k(y) g(w_1(x-y)) \, dy \\
&= \int_{x-\alpha}^{x-\beta} k(y)\,dy + \int_{x-\beta}^{\infty} m k(y)\,dy \\
&= \int_{x-\alpha}^{\infty} k(y)\,dy - \int_{x-\beta}^{\infty} (1-m) k(y)\,dy
\end{aligned}
\end{equation}
Taking the derivative with respect to $x$, we find
\begin{equation}
\frac{dw_2}{dx} = -k(x-\alpha) + (1-m)k(x-\beta)
\end{equation}
It follows from assumption iv. of \ref{hypothesis1} that $dw_2/dx$ has at most one zero-crossing. Hence, $w_2$ has at most one turning point. Since $w_2(-\infty)=m$ and $w_2(\infty)=0$, and $w_2$ is non-negative, there are only two cases.

First, $w_2$ has no turning point, and is monotone decreasing on $\mathbb R$. Then $w_2$ has a right inverse on $(0,m)$ satisfying $w_2(w_2^{-1}(p))=p$ for any $0<p<m$. Second, $w_2$ has a single turning point $x_0\in\mathbb R$. It follows that $w_2$ must be decreasing on $(x_0,\infty)$ and increasing on $(-\infty,x_0)$, with $w_2(x_0)\geq m$. In this case $w_2$ also has a well-defined right inverse on $(0,m)$, and furthermore $w_2^{-1}(p)>x_0$ for any $0<p<m$.

In either case, $w_2$ has a right inverse $w_2^{-1}$ on the interval $(0,m)$. Since $0<a<m$, we can take $c^*=\frac{1}{2}w_2^{-1}(a)$, so that $w_2(x)<a$ if and only if $x>2c^*$. Furthermore, since $||w_2||_\infty \leq b$, we have $w_2(x)\leq b$ almost everywhere. Thus,
\begin{equation} \label{gw2}
g(w_2(x)) = \begin{cases}
1 & x \leq 2c^* \\
0 & x > 2c^*
\end{cases}
\end{equation}
almost everywhere. Applying the convolution operator, we obtain
\begin{equation}
Q[w_2](x) = \int_{x-2c^*}^{\infty} k(y)\,dy = w_1(x-2c^*).
\end{equation}
\end{proof}

\begin{thm} \label{theorem1}
If $||w_2||_\infty\leq b$, then the sequence $(u_n)_{n=0}^{\infty}$ defined by
\begin{equation} \label{ptw}
u_{2n}(x) = w_1(x-2nc^*), \quad 
u_{2n+1}(x) = w_2(x-2nc^*)
\end{equation}
satisfies $u_{n+1}=Q[u_n]$ for all $n\geq 0$.
\end{thm}

\begin{proof}
By induction on $n$. For $n=0$, we have $u_0(x)=w_1(x)$ and $u_1(x)=Q[w_1](x)=w_2(x)$ by definition.

For the inductive step, assume $u_{2n}(x) = w_1(x-2nc^*)$ and $u_{2n+1}(x) = w_2(x-2nc^*)$ for some $n\geq 0$. By Lemma \ref{lemma1} and the translation invariance property of $Q$, we have
\begin{equation} \begin{aligned}
u_{2n+2}(x) = w_1(x-2nc^*-2c^*) = Q[w_2](x-2nc^*) = Q[u_{2n+1}](x)
\end{aligned} \end{equation}
 Likewise,
\begin{equation}
u_{2n+3}(x) = w_2(x-2nc^*-2c^*) = Q[w_1](x-2nc^*-2c^*) = Q[u_{2n+2}](x).
\end{equation}
\end{proof}

\begin{thm}  \label{theorem2}
If $||w_2||_\infty\leq b$, the sequence $(u_n)_{n=0}^{\infty}$ defined in Theorem \ref{theorem1} satisfies
\begin{equation}
\lim_{n\to\infty}\inf_{x<2nc}u_{2n}(x)=1, \quad \forall c\in(0,c^*)
\end{equation}
and
\begin{equation}
\lim_{n\to\infty}\inf_{x<2nc}u_{2n+1}(x)=m, \quad \forall c\in(0,c^*)
\end{equation}
and 
\begin{equation}
\lim_{n\to\infty}\sup_{x>nc}u_n(x)=0, \quad \forall c\in(c^*,\infty).
\end{equation}
\end{thm}

The next theorem concerns the spreading behavior of solutions to the IDE \eqref{q} with compactly supported initial data. We also assume the disperal kernel is compactly supported.

\begin{thm} Suppose $||w_2||_\infty \leq b$ and $k(x)$ has compact support. Assume the following hold:

\begin{enumerate}[i.)]
\item $c^*$ is positive,\
\item the initial condition $u_0$ has compact support;
\item the set $A=\{x\in\mathbb R:a\leq u_0(x)\leq b\}$  is connected and sufficiently large.
\end{enumerate}
 Then the sequence $(u_n)_{n=0}^{\infty}$ defined by $u_{n+1}=Q[u_n]$, $n\geq 0$, satisfies
\begin{equation}
\lim_{n\to\infty}\inf_{|x|<2nc}u_{2n}(x)=1, \quad \forall c\in(0,c^*)
\end{equation}
and
\begin{equation}
\lim_{n\to\infty}\inf_{|x|<2nc}u_{2n+1}(x)=m, \quad \forall c\in(0,c^*)
\end{equation}
and 
\begin{equation}
\lim_{n\to\infty}\sup_{|x|>nc}u_n(x)=0, \quad \forall c\in(c^*,\infty).
\end{equation}
\end{thm}

\begin{proof}
Let $(-\sigma,\sigma)$ be the support of $k$, for $0<\sigma<\infty$. Since $A$ is connected, we may also assume by the translation invariance property that $A=[-r,r]$ for some $r \geq 0$. Furthermore, since $u_0$ has compact support we must have $u_0(x)=0$ outside a bounded interval $(r^-,r^+) \supset A$, where $r^- < -r \leq r < r^+$.

Next, we will show that $0<u_0(x)<a$ for $x\in(r^-,r^+)\setminus A$. If we suppose to the contrary that $u_0(x)>b$ for some $x$ in this set, then suppose without loss of generality $x\in(r,r^+)$ such that $u_0(x)>b$. It follows by the intermediate value theorem that $u_0(x')=b$ for some $x' \in (x,r^+)$. But then $r\in A$, $x\notin A$, and $x'\in A$ with $r<x<x'$, violating the connectedness of $A$.

The preceding argument shows
\begin{equation}
g(u_0(x)) = \begin{cases}
1 & |x| < r \\
0 & |x| \geq r
\end{cases}
\end{equation}

To prove the theorem, we will show the spreading property is satisfied for all $r\geq2\sigma$. From $r\geq \sigma$ it follows,
\begin{equation}
u_1(x) = w_1(|x|-r)
\end{equation}
Then,
\begin{equation} g(u_1(x)) = \begin{cases}
m & |x|<r+\beta \\
1 & r+\beta\leq |x|\leq r+\alpha \\
0 & |x|>r+\alpha
\end{cases}
\end{equation}
where $\alpha$ and $\beta$ are defined as in Lemma \ref{lemma1}, and satisfy $-\sigma<\beta<\alpha<\sigma$. Note that $g(u_1(x))=g(Q[w](x))$ for all $x>-r-\beta$. Thus, $r\geq 2\sigma$ guarantees $r+\beta\geq\sigma$. It follows that
\begin{equation}
u_2(x) = Q[w_1](|x|-r) = w_2(|x|-r)
\end{equation}
Applying $g$, we get
\begin{equation}
g(u_2(x)) = \begin{cases}
1 & |x| \leq  r+2c^* \\
0 & |x| >  r+2c^*
\end{cases}
\end{equation}
Since $c^*>0$ we have $r+2c^*>r\geq\alpha$, so
\begin{equation}
u_3(x) = w_1(|x|-r-2c^*)
\end{equation}

The preceding argument can be repeated inductively to obtain
\begin{equation}
u_{2n+1}(x) = w_1(|x|-r-2nc^*)
\end{equation}
and
\begin{equation}
u_{2n+2}(x) = w_2(|x|-r-2nc^*)
\end{equation}
for all $n\geq 0$.
\end{proof}

\begin{rem}
This theorem indicates that a solution with proper compactly supported initial data coverges to translations of periodic traveling waves with profiles $w_1(x)$ and $w_2(x)$ in the positive direction and profiles $w_1(-x)$ and $w_2(-x)$ in the negative direction.
\end{rem}

\section{Examples}

In this section, we apply the results of Section 2 to three particular choices of dispersal kernel. We show that they satisfy Hypothesis \ref{hypothesis1}, and then provide an estimate of the wavespeed $c^*$.

\begin{ex}
The Laplace kernel,
\begin{equation}
k(x) = \frac{1}{2} e^{-|x|}
\end{equation}

The traveling waves are given by
\begin{equation}
w_1(x) =   \begin{cases} 
1 - \frac{1}{2}e^{x} & x \leq 0 \\
\frac{1}{2}e^{-x} & x > 0
\end{cases}
\end{equation}
and
\begin{equation}
w_2(x) = \int_{-\infty}^\winvb mk(x-y)\,dy + \int_{\winvb}^\winva k(x-y)\,dy
\end{equation}
with
$$ w_1^{-1}(p) = \begin{cases} -\log(2p) & p\leq \frac{1}{2} \\ \log(2-2p) & p > \frac{1}{2} \end{cases} $$
Then
$$ w_2(x) = \begin{cases}
\int_{-\infty}^{-\log(2b)} \frac{m}{2}e^{-|x-y|}\,dy + \int_{-\log(2b)}^{-\log(2a)} \frac{1}{2}e^{-|x-y|} \,dy & a<\frac{1}{2}, b < \frac{1}{2} \\
\int_{-\infty}^{\log(2-2b)} \frac{m}{2}e^{-|x-y|} \,dy + \int_{\log(2-2b)}^{-\log(2a)} \frac{1}{2}e^{-|x-y|} \,dy & a<\frac{1}{2}, b \geq \frac{1}{2} \\ 
\int_{-\infty}^{\log(2-2b)} \frac{m}{2}e^{-|x-y|} \,dy + \int_{\log(2-2b)}^{\log(2-2a)} \frac{1}{2}e^{-|x-y|} \,dy & a\geq\frac{1}{2}, b \geq \frac{1}{2} \\
\end{cases} $$
Now make the substitution $y'=x-y$.
$$ w_2(x) = \begin{cases}
\int_{x+\log(2b)}^{\infty} \frac{m}{2}e^{-|y'|}\,dy' + \int_{x+\log(2b)}^{x+\log(2a)} \frac{1}{2}e^{-|y'|} \,dy' & a<\frac{1}{2}, b < \frac{1}{2} \\
\int_{x-\log(2-2b)}^{\infty} \frac{m}{2}e^{-|y'|} \,dy' + \int_{x-\log(2-2b)}^{x+\log(2a)} \frac{1}{2}e^{-|y'|} \,dy' & a<\frac{1}{2}, b \geq \frac{1}{2} \\ 
\int_{x-\log(2-2b)}^{\infty} \frac{m}{2}e^{-|y'|} \,dy' + \int_{x-\log(2-2b)}^{x-\log(2-2a)} \frac{1}{2}e^{-|y'|} \,dy' & a\geq\frac{1}{2}, b \geq \frac{1}{2} \\
\end{cases} $$

We will now split into further cases.

Case 1: $a<b<\frac{1}{2}$. Then
$$ w_2(x) = \int_{x+\log(2b)}^{\infty} \frac{m}{2}e^{-|y|}\,dy + \int_{x+\log(2b)}^{x+\log(2a)} \frac{1}{2}e^{-|y|} \,dy $$
$$ = $$


\end{ex}

\begin{ex} Consider the Gaussian kernel with mean $0$ and variance $1$ given by
$$ k(x) = \frac{1}{\sqrt{2\pi}} e^{-\frac{x^2}{2}} $$
$k$ is symmetric, strictly increasing on $(-\infty,0)$ and strictly decreasing on $(0,\infty)$, hence conditions i.-iii. of Hypothesis \ref{hypothesis1} are satisfied. For condition iv., let $y \in \mathbb R$ and $\mu\in(0,1)$. It can be shown that $k(x) - \mu k(x-y) = B(x)(1-Ce^{-xy})$ where $B(x)=e^{-\frac{x^2}{2}}/\sqrt{2\pi}$ is a strictly positive function and $C>0$. This quantity decreases monotonically from $1$ to $-\infty$ with a unique zero-crossing at $x=\frac{y}{2}-\frac{\ln \mu}{y}$. Thus, Hypothesis \ref{hypothesis1} is satisfied.

The periodic traveling wave solutions $w_1(x)$ and $w_2(x)$ are given by
\begin{equation}
w_1(x) = \frac{1}{2} - \frac{1}{2}\erf\left(\frac{x}{\sqrt{2}}\right)
\end{equation}
and
\begin{equation}
w_2(x)=  \frac{m}{2} + \frac{1-m}{2} \erf \left( \frac{x-\beta}{\sqrt 2}\right) - \frac{1}{2} \erf \left( \frac{x-\alpha}{\sqrt2} \right)
\end{equation}
where $\alpha=\sqrt{2}\erf^{-1}\left(1-2a\right)$, and $\beta=\sqrt{2}\erf^{-1}\left(1-2b\right)$, and $\erf$ is the error function defined by $\erf(x) =\frac{2}{\sqrt\pi}\int_0^x e^{-y^2}\,dy$.

$w_2$ has a unique global maximum at $x^*=\frac{\alpha+\beta}{2} + \frac{1}{\alpha-\beta}\ln\left(1-m\right)$. Thus, by Theorem \ref{theorem1}, $w_1$ and $w_2$ are a periodic traveling wave solution if $w_2(x^*)\leq b$.
\end{ex}

%\begin{ex} The Laplace kernel is given by
%$$ k(x) = \frac{1}{2} e^{-\frac{1}{2}|x|} $$
%
%It can be checked that $k(x)$ satisfies all of Hypothesis \ref{hypothesis1}. To show part iv., let $\mu\in(0,1)$, and assume without loss of generality $y>0$. (If $y=0$ then $f(x)$ is simply $k(x)$ rescaled, hence strictly positive.) For $y>0$, we have
%$$ f(x) = k(x) - \mu k(x-y) = \frac{1}{2} \begin{cases}
%e^x - \mu e^{x-y} & \text{if } x < 0 \\
%e^{-x} - \mu e^{x-y} & \text{if } 0 \leq x < y \\
%
%e^{-x} -\mu e^{-(x-y)} & \text{if } x \geq y
%\end{cases} $$
%$f$ is continuous, satisfies $f(-\infty)=f(\infty)=0$, is increasing on $(-\infty,0)$ and decreasing on $(0,\alpha)$. If $e^y\geq \mu$, then $f$ is decreasing on $(y,\infty)$, so $f$ has no zero-crossings Otherwise, if $e^y< \mu$, then $f$ is increasing on $(y,\infty)$ and has exactly one zero-crossing. In both cases, condition iv. is satisfied.
%
%We have three cases:
%
% If $a<\frac{1}{2}$ and $b<\frac{1}{2}$, then
%$$ \begin{aligned}
%w_2(-x) &= n_{2}K\left(x-\frac{1}{\alpha}\ln\left(2a\right)\right)-\left(1-m\right)K\left(x-\frac{1}{\alpha}\ln\left(2b\right)\right) \\
%&= \begin{cases}
%\left(\frac{1}{4a}-\frac{1-m}{4b}\right)e^{\alpha x} & x < \frac{1}{\alpha}\ln\left(2a\right) \\
%1-ae^{-\alpha x}-\frac{1-m}{4b}e^{\alpha x} &  \frac{1}{\alpha}\ln\left(2a\right) \leq x < \frac{1}{\alpha}\ln\left(2b\right) \\
%m+\left((1-m)b-a\right)e^{-\alpha x} &  x \geq \frac{1}{\alpha}\ln\left(2b\right)
%\end{cases} 
%\end{aligned} $$
%If $b-(1-m)a\geq 2ab$, then $c^*$ is given by $c^*=\frac{1}{2\alpha}\ln\left(\frac{b-(1-m)a}{4a^2b}\right)$.
%\end{ex}

\begin{ex}
Consider the uniform dispersal kernel given by
\begin{equation}
k(x) = \begin{cases}
\frac{1}{2} & |x|\leq 1 \\
0 & |x| > 1
\end{cases} \end{equation}

Then $w_1$ is given by
\begin{equation}  \label{w1}
\begin{aligned}
w_1(x) 
= \begin{cases}
1, & x \in (-\infty, -1), \\
\frac{1}{2}-\frac{1}{2}x, & x \in [-1, 1] ,\\
0, & x \in (1, \infty),
\end{cases}
\end{aligned} \end{equation}
with inverse $w_1^{-1}(p)=1-2p$ for $0<p<1$. Let $\alpha=1-2a$ and $\beta=1-2b$. Then
\begin{equation} \label{w2}
\begin{aligned}
w_2(x) 
= \begin{cases}
m,
& x \in (-\infty, \beta-1), \\
\frac{1-m}{2}x + m + b - mb,
& x \in [\beta - 1, 
 \alpha- 1), \\
-\frac{m}{2}x +m+b- mb-a,
& x \in [\alpha - 1, \beta + 1), \\
-\frac{1}{2} x-a+1,
& x \in [\beta + 1, \alpha + 1], \\
0,
& x \in (\alpha+1,\infty).
\end{cases}
\end{aligned} \end{equation}

Observe that $w_2$ has a global maximum at $x=\alpha-1$ so that $||w_2||_\infty=w_2(\alpha-1)=m+(b-a)(1-m)$. By Theorem \ref{theorem2}, the pair $w_1$ and $w_2$ are a solution to equation \eqref{ptw} if $m-a<m(b-a)$.

We can also explicitly calculate the speed of the wave given by
\begin{equation} \label{c}
c^* = \begin{cases}
1 - 2a & \text{if } a \leq b/2, \\
1 -b + \frac{b - 2a}{m} & \text{if }a > b/2.
\end{cases}
\end{equation}
\end{ex}
\begin{rem}
$w_1(x)$ is positive for $x<1$ and zero for $x\geq1$, and $w_2(x)$ is positive for $x<2-2a$ and zero for $x\geq 2-2a$. Thus, $\eqref{q}$ has a traveling wave with wave profiles $w_1(x)$ and $w_2(x)$, intermediate wave speeds $c_1=1-2a$ and $c_2=2c^*-c_1$,  and average wave speed  $c^*$.  It is easily seen that $c_1=c_2$ if $a \leq b/2$, and $|c_1-c_2|=(2\alpha-\beta)(1-\frac{1}{m})>0$ if $a>b/2$. So for $a>b/2$, the traveling wave is periodic with two different intermediate wave speeds. Furthermore, the difference between these two intermediate speeds is increasing in $a$, decreasing in $b$, and increasing in $m$. This behavior is illustrated with two difference choices of parameters in Figure \ref{fig:wavespeed}.
\end{rem}


The regions in the parameter space where oscillating spreading speed exists can be determined as follows: for any fixed choice of $(n_1,n_2)$, with $0<n_1<n_2$, let $R$ be the set of pairs $(a,b)\in\mathbf R^2$ such that the hypothesis of Theorem 2.1 holds. Then $R$ is a triangle in the $a$-$b$ plane with endpoints at $(0,n_2)$, $(n_1,n_1)$, and $(n_1,n_2)$, depicted in Figure \ref{fig:phaseportrait}. The line $b=2a$ partitions $R$ into two non-empty sets $R_1=\{(a,b)\in R:a\leq b/2\}$ and $R_2=\{(a,b)\in R:a> b/2\}$ such that the traveling has constant speed if $(a,b)\in R_1$ and oscillating speed if $(a,b)\in R_2$.


\begin{thebibliography}{9}
\bibitem{all}  W. C. Allee. 1931. Animal Aggregations. A Study on
General Sociology. University of Chicago Press, Chicago, IL.


%\bibitem{htw90} D. P. Hardin, P. Tak$\acute{\mbox{a}}\check{\mbox{c}}$, and G. F. Webb. 1990. Dispersion population %models discrete in time and
%continuous in space. J. Math. Biol. {\bf 28}: 1-20.

\bibitem{hh} A. Hastings and K. Higgins. 1994. Persistence of transients in spatially
structured ecological models. Science {\bf 263}: 1133-1136.

%\bibitem{hz} S.-B. Hsu and X.-Q. Zhao. 2008. Spreading speeds and traveling waves for nonmonotone integrodifference equations. SIAM J. Math. Anal. {\bf} 40: 776–789


\bibitem{ks} M. Kot and W. M. Schaffer. 1986.  Discrete-time growth-dispersal models.
Math. Biosci. {\bf 80}: 109-136.


\bibitem{kot89} M. Kot. 1989. Diffusion-driven period doubling bifurcations.
Biosystems {\bf 22}: 279-287.


\bibitem{kot92} M. Kot. 1992. Discrete-time traveling waves:
Ecological examples. J. Math. Biol. {\bf 30}: 413-436.


\bibitem{kot1} M. Kot, M. A. Lewis, and P. van den Driessche. 1996. Dispersal data and the spread of invading
organisms. Ecology {\bf 77}: 2027-2042.


%\bibitem{kot2} M. Kot, J. Medlock, T. Reluga, and D. B. Walton. 2004. Stochasticity, invasions, and branching random walks. Theor. Popul. Biol. {\bf 66}: 175-184.


%\bibitem{li09} B. Li, M. A. Lewis, and H. F. Weinberger. 2009.  Existence of traveling waves for integral recursions with nonmonotone growth functions. J. Math. Biol. {\bf 58}: 323-338.


\bibitem{kotbook} M. Kot. 2001. Elements of Mathematical Ecology.
Cambridge University Press. Cambridge, United Kingdom.

\bibitem{lui82a} R. Lui. 1982. A nonlinear integral operator arising from a model in population
genetics. I. Monotone initial data. SIAM. J. Math. Anal. {\bf 13}:
913-937.

\bibitem{lui82b} R. Lui. 1982. A nonlinear integral operator arising from a model in population
genetics. II. Initial data with compact support. SIAM. J. Math.
Anal. {\bf 13}: 938-953.

\bibitem{lui83} R. Lui. 1983. Existence and stability of traveling wave solutions of a nonlinear
integral operator. J. Math. Biol. {\bf 16}:199-220.

\bibitem{lut}
F. Lutscher. 2019. Integrodifference Equations in Spatial Ecology.  Springer.


\bibitem{nkl} M. Neubert, M. Kot, and M. A. Lewis. 1995. Dispersal and pattern formation in a
discrete-time predator-prey model. Theor. Pop. Biol. {\bf 48}
: 7-43.

\bibitem{otto} G. Otto. 2017. Non-spreading Solutiona in a Integro-Difference Model Incorporating Allee and Overcompensation Effects. Ph. D thesis, University of Louisville.

\bibitem{slatkin} M. Slatkin. 1973. Gene flow and selection in a cline.
Genetice {\bf 75}: 733-756.



\bibitem{pnas} L. L. Sullivan, B. Li, T. E. X. Miller, M. G. Neubert, and A. K. Shaw. 2017.
Density dependence in demography and dispersal generates fluctuating invasion speeds. Proc. Natl. Acad. Sci. USA {\bf
114}: 5053-5058.


\bibitem{wang} M. H. Wang, M. Kot, and M. G. Neubert. 2002. Integrodifference equations, Allee effects, and
invasions. J. Math. Biol. {\bf 44}: 150-168.

\bibitem{w78} H. F. Weinberger. 1978. Asymptotic behavior of a model in  population genetics,
in Nonlinear Partial Differential Equations  and Applications, ed.
J. M. Chadam. Lecture Notes in Mathematics {\bf 648}: 47-96.
Springer-Verlag, Berlin.

\bibitem{wein82} H. F.  Weinberger. 1982. Long-time beahvior of a class of biological models. SIAM. J.
Math. Anal. {\bf 13}: 353-396.

\end{thebibliography}

\end{document}









\end{document}
